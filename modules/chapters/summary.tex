\begin{center}
	\Large{
		\textbf{Аннотация}
	}
\end{center}

Целью данной курсовой работы является приобретение навыков расчёта 
и проектирования микроэлектронных устройств на базе систем «Multisim» 
и «NI ELVIS», использование справочной литературы, 
оформление технической документации.

При выполнении курсовой работы разрабатывается функциональная 
и принципиальная схемы усилителя, рассчитывается, 
и выбираются входящие в его состав компоненты, 
определяющие уточнённые статические и динамические параметры 
электронного устройства.

На заключительном этапе проектирования оформляется пояснительные 
записки и чертёж принципиальной электронной схемы усилителя, 
выполненный в соответствии с Единой системы конструкторской 
документации (ЕСКД). 

\\
\\
\\
\\
\\
\\
\\
\\
\\
\begin{center}
	\Large{
		\textbf{Summary}
	}
\end{center}

The purpose of this course project is to acquire the skill 
of calculating and designing microelectronic devices based 
on the systems "Multisim" and "NI ELVIS", the use of reference 
literature, the design of technical documentation.

During the implementation of the planned project, electrical 
and circuit diagrams of the amplifier are developed, calculated, 
and selected from its constituent components, which determine 
the specified static and dynamic parameters of the electronic device.

At the final design stage, explanatory notes and a schematic 
diagram of the amplifier circuit are drawn up, made in accordance 
with the Unified Design System (ESKD).
