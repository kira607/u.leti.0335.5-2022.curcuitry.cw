\section*{Техническое задание}
\addcontentsline{toc}{section}{Техническое задание}

Усилитель переменного тока должен быть реализован 
на маломощных интегральных операционных усилителях (ОУ) 
с выходной мощностью не более 100 мВт. 
Для обеспечения большой выходной мощности всего усилителя 
переменного тока, отдаваемой в нагрузку (10 - 15 Вт), 
на его выходе может быть использован выходной каскад на 
дискретных компонентах. Выходные транзисторы этого 
каскада могут устанавливаться на теплоотвод.

\begin{table}[H]
    \centering
    \begin{tabular}{|c|c|}
        \hline
        Коэффициент усиления в полосе пропускания $ K_U $ & 1000 \\
        \hline
        Нижняя граничная частота полосы пропускания $ f_н $, Гц & 200 \\
        \hline
        Верхняя граничная частота $ f_в $, кГц, не менее & 20 \\
        \hline
        Входное сопротивление $ R_{вх} $, кОм & 200 \\
        \hline
        Постоянное напряжение помехи на выходе $ U_{п.вых} $, В, не более & 1.5 \\
        \hline
        Максимальный ток нагрузки $ I_н $, А, не менее & 1.4 \\
        \hline
        Максимальное выходное напряжение $ U_{вых} $, В, не менее & 10 \\
        \hline
    \end{tabular}
    \caption{Техническое задание. Параметры усилителя переменного тока. Вариант 16}
    \label{tab:technical_task}
\end{table}
